\documentclass[10pt, lettersize, journal, onecolumn]{IEEEtran}

%\mathtoolsset{showonlyrefs = true} BETTER NOT
%\usepackage[a4paper,width=185mm,top=20mm,bottom=20mm]{geometry}
\usepackage[T1]{fontenc}
\usepackage[backend=biber]{biblatex}
\usepackage[caption=false,font=normalsize,labelfont=sf,textfont=sf]{subfig}
\usepackage[makeroom]{cancel}
\usepackage{MnSymbol}
\usepackage{algorithmic}
\usepackage{algorithm}
\usepackage{amsmath, amsfonts, mathtools}
\usepackage{array}
\usepackage{ascii}
\usepackage{colortbl}
\usepackage{color}
\usepackage{comment}
\usepackage{csquotes}
\usepackage{float}
\usepackage{graphicx}
\usepackage{lipsum}
\usepackage{listings}
\usepackage{multicol}
\usepackage{setspace}
\usepackage{textcomp}
\usepackage{url}
\usepackage{wrapfig}
\usepackage{xcolor}

\DeclareGraphicsExtensions{.png, .pdf, .pgf}
\addbibresource{./bibliography.bib}
\definecolor{darkred}{rgb}{0.8,0,0}
\graphicspath{{../results/}{../immagini/}}
\hyphenation{op-tical net-works semi-conduc-tor IEEE-Xplore}
\setlength{\columnsep}{1cm}

\title{Generalization of NLIN model - Approximation for time integrals}
\author{Francesco Lorenzi  \vspace{2pt}\\  {\footnotesize Dipartimento di Ingegneria dell'Informazione\\ \vspace{-5pt} Università degli studi di Padova}}

\begin{document}
	\maketitle
	The approximation in equation (20) (corrected in errata, eq.(3)) relates the pulse time evolution with its spectrum:

	\begin{equation}\label{eq:papoulis}
		g^{(0)}(z, t) \approx \sqrt{\frac{i}{2\pi \beta_2 z}} \exp\left[-\frac{it^2}{2 \beta_2 z}\right] \hat{g}\left(0, \frac{t}{\beta_2 z}\right)	
	\end{equation}
	the computation of this approximation is done in Papoulis \cite{Papoulis_1994}.
	To clarify the expression, the passages are derived below.
	
	We know that both coupled NLS equations have linear propagators $\mathbf{U}_A(z)$ and $\mathbf{U}_B(z)$. Both propagators have a chromatic dispersion part, whereas $\mathbf{U}_B(z)$ contains also a group velocity term. Focusing on dispersion, we use the symbol $\mathbf{U}(z)$ to represent the dispersion propagator
	\begin{equation}
		\mathbf{U}(z) = \exp\left[-i \dfrac{\beta_2}{2}z\dfrac{\partial^2}{\partial t^2} \right]
	\end{equation}
	which corresponds to the following frequency response 
	\begin{equation}
		\hat{\mathbf{U}}(z) = \exp\left[i \dfrac{\beta_2}{2}z\omega^2 \right].
	\end{equation}
	The last frequency response is a \textit{Quadratic Phase Filter} (QPF): an all-pass filter with quadratic phase relationship. 
	Remember the Gaussian function Fourier transform and antitranform pair
	\begin{equation}
		\exp (-st^2)  \xrightarrow{\mathcal{F}}  \sqrt{\dfrac{\pi}{s}}\exp\left(-\dfrac{\omega^2}{4 s}\right).
	\end{equation}
	The multiplication by the propagator in the frequency domain corresponds to a convolution in the time domain by a Gaussian
	\begin{equation}
		g^{(0)}(z, t) = \sqrt{\dfrac{i}{2\pi \beta_2 z}} \int_{-\infty}^{\infty}  g(0, \tau) \exp\left[-i\dfrac{(t-\tau)^2}{2\beta_2 z}  \right] d\tau
	\end{equation}
	This particular convolution is a \textit{Fresnel transform}, of the form
	\begin{equation}
		g^{(0)}(z, t) = \sqrt{\dfrac{\alpha}{i\pi}} \int_{-\infty}^{\infty}  g(0, \tau) \exp\left[-i \alpha(t-\tau)^2 \right] d\tau
	\end{equation}
	with
	\begin{equation}
		\alpha = \dfrac{1}{2 \beta_2 z} 
	\end{equation}
	By expanding the square in the exponential, 
	\begin{equation}
		g^{(0)}(z, t) = \sqrt{\dfrac{i}{2\pi \beta_2 z}} \exp\left[-i\dfrac{t^2}{2\beta_2 z}\right] \int_{-\infty}^{\infty}  g(0, \tau) \exp\left[-i\dfrac{t^2- 2t \tau +\tau^2}{2\beta_2 z}  \right] d\tau
	\end{equation}
	we can select terms which 
	
	The approximation written in \ref{eq:papoulis} is possible when the following expression is 
	
	In terms of fiber length and dispersion length, for a Gaussian pulse we have
	
	
	
	A general theory for 
	In presence of small dispersion, the approximation is accurate if the pulse shape is an eigenfunction of the chirp operator.
	
	
	
	
	\printbibliography
\end{document}